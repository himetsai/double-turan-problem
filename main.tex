\documentclass[12pt]{report}
\usepackage{amsmath, amssymb, amsthm}
\usepackage{geometry}
\usepackage{graphicx}
\usepackage{hyperref}
\usepackage[noabbrev,capitalize]{cleveref}
\usepackage{listings}
\usepackage{enumitem}
\usepackage{titlesec}
\usepackage{setspace}
\usepackage{mathtools}
\usepackage{xcolor}
\usepackage{tikz}
\usepackage{tikz-cd}
\usepackage[colorinlistoftodos]{todonotes}
\usepackage{algorithm}
\usepackage{algpseudocode}
\titleformat{\chapter}[hang]{\normalfont\huge\bfseries}{\thechapter}{1em}{}
\usetikzlibrary{automata,positioning}

\definecolor{crimson}{HTML}{DC143C}

% Page layout
\geometry{a4paper, margin=1in}
\setlength{\parskip}{5pt}
\setlength{\parindent}{0pt}
\textheight 9in \textwidth 6.5in
\renewcommand{\baselinestretch}{1.1}

\hypersetup{
  colorlinks=true,
  linkcolor=crimson,
  urlcolor=crimson,
  citecolor=crimson
}

% Theorem Environments
\newtheorem{theorem}{Theorem}[chapter]
\newtheorem{lemma}[theorem]{Lemma}
\newtheorem{claim}{Claim}[theorem]
\newtheorem{conjecture}{Conjecture}[chapter]
\newtheorem{definition}[theorem]{Definition}
\newtheorem{corollary}[theorem]{Corollary}
\newtheorem{proposition}[theorem]{Proposition}

\newcommand{\Var}{\mathrm{Var}} 
\newcommand{\Cov}{\mathrm{Cov}} 
\newcommand{\Bias}{\mathrm{Bias}} 
\newcommand*{\Z}{\mathbb{Z}} 
\newcommand*{\Q}{\mathbb{Q}} 
\newcommand*{\R}{\mathbb{R}} 
\newcommand*{\C}{\mathbb{C}} 
\newcommand*{\N}{\mathbb{N}} 
\newcommand*{\prob}{\mathds{P}} 
\newcommand*{\E}{\mathds{E}} 
\newcommand*{\F}{\mathds{F}}
\newcommand*{\ex}{\textnormal{ex}}
\newcommand*{\dex}{\textnormal{ex}_2}
\newcommand*{\con}{\mathcal{C}}

% Title Page
\title{Double Turán Problem}

\begin{document}

\maketitle

% \begin{abstract} Your abstract here. Provide a concise summary of your thesis, including the problem statement, methodology, main results, and conclusions. \end{abstract}

% \chapter*{Acknowledgments} Thank your advisor, collaborators, and anyone who supported your research.

\tableofcontents

\chapter{Introduction}

Let $F$ be a graph with at least one edge. A graph $G$ is \textit{$F$-free} if $G$ does not contain $F$ as a subgraph. The fundamental question in extremal graph theory is to determine the maximum number of edges in an $n$-vertex $F$-free graph. These maxima, denoted $\ex(n, F)$, are referred to as the \textit{extremal numbers} or \textit{Turán numbers} for $F$.

In this thesis, we investigate a closely related problem which we refer to as the \textit{double Turán problem}. Let $G_1, G_2, \ldots, G_m$ be graphs on the same vertex set of size $n$. We are interested in determining the maximum sum of edges over $m$ graphs $G_1, G_2, \ldots, G_m$ whose pairwise intersection is $F$-free. We denote this quantity as $\dex(m, n, F)$ and refer to an $F$ in the intersection of two of the graphs $G_i$ as a \textit{double $F$}. 

\section{Definitions and Notation}

Denote the set of first $n$ positive integers as $[n] = \{1, 2, \ldots, n\}$. Given a set $X$, we denote $2^X$ as the power set of $X$.

Let $G = (V, E)$ be a graph. Let $V(G) = V$ denote the vertex set and $E(G) = E$ denote the edge set of $G$. We note by $v(G) = |V|$ the number of vertices and $e(G) = |E|$ the number of edges in $G$. For vertex $v \in V(G)$, we denote by $N_G(v) = \{u \in V(G) : \{u, v\} \in E(G)\}$ the neighborhood of $v$.

Given graphs $G_1, \ldots, G_m$ on some vertex set $V$, we denote $G_{i_1, \ldots, i_k}$ as graph on $V$ with edge set $E(G_{i_1, \ldots, i_k}) = \bigcap_{\alpha = 1}^k E(G_{i_\alpha})$. Given two graphs $G_1, G_2$, we denote $G_1 \cup G_2$ as the graph on $V(G_1) \cup V(G_2)$ with edge set $E(G_1 \cap G_2) = E(G_1) \cup E(G_2)$. Let $s$

In this thesis, we reserve $n$ to denote the number of vertices in a graph. We call a $n$-vertex complete graph $K_n$, and a complete bipartite graph $K_{a, b}$, where $a, b$ are the size of its parts. We denote $P_n$ as a path with $n$ edges, and $C_n$ as a cycle with $n$ edges. Given graph $G, H$, define $G + H$ as the graph fully connecting $G, H$, i.e. $V(G + H) = V(G) \cup V(H)$ and $E(G + H) = E(G) \cup E(H) \cup \{\{u, v\} : u \in V(G), v \in V(H)\}$.

Given graphs $G$ and $F$, we say that $G$ is $F$-free if $G$ does not contain $F$ as a subgraph. We denote $\ex(n, F)$ to be the maximum  possible number of edges an $F$-free graph on $n$ vertices, and we call a $F$-free graph achieving this maximum an extremal graph for $F$. Given graphs $G_1, \ldots, G_m$ on the same set of vertices and $F$, we say that $G_1, \ldots, G_m$ are pairwise $F$-free if $E(G_i) \cap E(G_j)$ does not contain $F$ for $i \neq j$. Let $v$ be a vertex from $G_1, G_2, \ldots, G_m$. Unless otherwise specified, we denote $d(v)$ as the sum of degree of $v$ over all $G_i$.

\section{Problem Statement}

Let $\dex(m, n, F)$ be the maximum possible number of edges that $m$ pairwise $F$-free graphs on $n$ vertices can have. Our goal is to determine $\dex(m, n, F)$ for different forbidden graphs $F$. A trivial construction with $G_1 = K_n$ and $G_2, \ldots, G_m$ to be extremal graphs for $F$ yields the lower bound $\dex(m, n, F) \geq \binom{n}{2} + (m - 1)\ex(n, F)$. In this work, we use this bound as a benchmark to either show its tightness or to improve it.

Additionally, we are also interested in a more restrictive version where $G_1, \ldots, G_m$ are induced subgraphs of $G_1 \cup \cdots \cup G_m$. Let $\dex^*(m, n, F)$ as the maximum possible number of edges that $m$ pairwise $F$-free graphs on $n$ vertices can have, with the constraint that each graph is an induced subgraph of their union. A trivial construction with $G_1 = \cdots = G_m$ to be extremal graphs for $F$ yields the lower bound $\dex^*(m, n, F) \geq m \cdot \ex(n, F)$. This is the benchmark we use to determine $\dex^*(m, n, F)$. Similar to the non-induced case, we will use this bound as a benchmark and base our work on it.

\chapter{Induced Version}

In this chapter, we investigate the case where $G_1, \ldots, G_m$ are induced subgraphs of $G_1 \cup \cdots \cup G_m$ and are pairwise $F$-free, for some specified $F$. Unless otherwise specified, when we say $G_1, \ldots, G_m$ are induced subgraph, we mean that they are induced subgraphs of $G_1 \cup \cdots \cup G_m$.

The following lemma shows that the problem can be redueced to only two graphs.

\begin{lemma}\label{lem:induce-reduce}
  Let $n, m, k \geq 1$ such that $2 \leq k \leq m$, $F$ be some graph, and $G_1, \ldots, G_m$ be pairwise $F$-free induced subgraphs on $n$ vertices. Then
  \[
    \dex^*(m, n, F) \leq \frac{m}{k} \cdot \dex^*(k, n, F).
  \]
  Moreover, if $\sum_{i = 1}^k e(G_i) = \dex^*(k, n, F)$ only if $G_1 = \cdots = G_k$, then $\sum_{i = 1}^m e(G_i) = \dex^*(m, n, F)$ only if $G_1 = \cdots = G_m$ and $\dex^*(m, n, F) = \frac{m}{k} \cdot \dex^*(k, n, F)$.
\end{lemma}

\textcolor{red}{Not putting equality because I'm unsure if a construction for $k$ subgraphs can always generalize to $m$ subgraphs. For example, if $F = K_3$ and $n$ is odd, the $G_1 = K_{\left\lceil\frac{n - 1}{2}\right\rceil, \left\lfloor\frac{n - 1}{2}\right\rfloor}$ and $G_2 = K_{\left\lceil\frac{n - 1}{2}\right\rceil, \left\lfloor\frac{n - 1}{2}\right\rfloor} + K_1$ construction cannot be generalized to $m = n + 1$ subgraphs.}

\begin{proof}
  Let $G_1, \ldots, G_m$ be induced subgraphs of $G_1 \cup \cdots \cup G_m$ with $E(G_i) \cap E(G_j)$ not containing $F$ for $i \neq j$. Put $G_{i + m} = G_i$ for all $i \in [m]$. Then
  \[
    \sum_{i = 1}^m e(G_i) = \frac{1}{k}\sum_{i = 1}^m [e(G_i) + \cdots + e(G_{i + k - 1})] \leq \frac{1}{k}\sum_{i = 1}^m \dex^*(k, n, F) = \frac{m}{k} \cdot \dex^*(k, n, F),
  \]
  which establishes the upper bound.

  Suppose $\sum_{i = 1}^k e(G_i) = \dex^*(k, n, F)$. By assumption $G_1 = \cdots = G_k$, so $e(G_i) = \dex^*(k, n, F)/k$ for $1 \leq i \leq k$. Hence, the construction $G_1 = \cdots = G_m$ meets the upperbound. On the other hand, if $G_1 \neq G_2$ then $\sum_{i = 1}^k e(G_i) < \dex^*(k, n, F)$. Since $\sum_{i = 1}^k e(G_{i + j}) \leq \dex^*(k, n, F)$ for all $j \geq 1$, we have $\sum_{i = 1}^m e(G_i) < \frac{m}{k} \cdot \dex^*(k, n, F)$. Thus the extremal condition is met only when $G_1 = \cdots = G_m$.
\end{proof}

\cref{lem:induce-reduce} allows us to reduce the problem to the case for two subgraphs $G_1, G_2$. Let $C = V(G_1) \cap V(G_2)$, $c = |C|$, $d = |V(G_1) \backslash A|$, and $n - c - d = |V(G_2) \backslash C|$. Note that $c, d \in \Z_{\geq 0}$. Since $G_1, G_2$ are induced subgraphs of $G_1 \cup G_2$, we have $G_1[C] = G_2[C] = G_{1, 2}$. But then $G_{1, 2}$ is $F$-free, so $e(G_1[C]) = e(G_2[C]) \leq \ex(c, F)$. Given $c, d$, the optimal construction to maximize the number of edges over $G_1, G_2$ is thus putting $G_{1, 2}$ as an extremal graph for $F$ on $c$ vertices and connect all edges that are not induced in $A$. This yields the inequality
\[
  e(G_1) + e(G_2) \leq \binom{d}{2} + \binom{n - c - d}{2} + (n - c)c + 2\ex(c, F).
\]
But then notice that $\binom{n - c}{2} > \binom{d}{2} + \binom{n - c - d}{2}$ for $0 < d < n - c$. This implies our construction is optimized when $d = 0$ or $d = n - c$, that is, to let $G_2$ contain $G_1$ or the other way around. Hence, we may assume $d = 0$ and define the construction function as
\[
  \con(n, c, F) \coloneq \binom{n - c}{2} + (n - c)c + 2\ex(c, F),
\]
i.e. the number of edges over two induced graphs in the above construction. Since $e(G_1) + e(G_2) \leq \con(n, c, F)$ for some given $c$, we have the following:
\begin{lemma}\label{lem:optimize-con}
  Let $F$ be some graph. For $n \geq 1$,
  \[
    \dex^*(2, n, F) = \max_{0 \leq c \leq n} \con(n, c, F).
  \]
  Moreover, let $G_1, G_2$ be induced pairwise $F$-free subgraphs and $c_{max}$ be some maximizer of $\con(n, c, F)$. Then $e(G_1) + e(G_2) = \dex^*(2, n, F)$ only if $G_1, G_2$ are the construction described by $\con(n, c_{max}, F)$.
\end{lemma}

The problem is now reduced to maximizing $\con$ over $c$. In particular, $\con(n, n, F)$ gives our benchmark construction of $G_1 = G_2$ being the extremal graphs for $F$ on $n$ vertices. For $0 \leq k \leq c \leq n$, define
\[
  \Delta_k \con(n, c, F) \coloneq \con(n, c, F) - \con(n, c - k, F) = \frac{1}{2}k(k - 2c + 1) + 2[\ex(c, F) - \ex(c - k, F)]
\]
and denote $\Delta \con = \Delta_1 \con$. Most of the work in this section will show that the maximum of $\con$ happens when $c \geq n - k$ by proving that $\Delta_k \con(n, c, F) > 0$ for all $c \leq n - k$.

\begin{lemma}\label{lem:induce-cond}
  Let $n, c_0 \geq 1$, $m \geq 2$, and $F$ be some graph. If $\con(n, c, F)< 2 \cdot \ex(n, F)$ for $0 \leq c < c_0$ and $\ex(c, F) - \ex(c - 1, F) > \frac{c - 1}{2}$ for $c_0 \leq c \leq n$, then
  \[
    \dex^*(m, n, F) = m \cdot \ex(n, F)
  \]
  and the extremal condition is met if and only if all $m$ induced pairwise $F$-free subgraphs are equal and extremal graphs for $F$.
\end{lemma}

\textcolor{red}{This should be if and only if and I will strengthen it shortly.}

\begin{proof}
  By \cref{lem:induce-reduce} and \cref{lem:optimize-con}, it suffices to show $\con(n, c, F)$ has a unique maximum of $2\ex(n, F)$ at $c = n$. We may assume $c \geq c_0$ by assumption. Suppose $c < n$. Since $\ex(c, F) - \ex(c - 1, F) > \frac{c - 1}{2}$,
  \[
    \Delta \con(n, c, F) = - c + 1 + 2[\ex(c, F) - \ex(c - 1, F)] > 0.
  \]
  Thus, $\con$ is strictly increasing with respect to $c$ for $c \geq c_0$, so $\con$ has a unique maximum of $2 \cdot \ex(n, F)$ at $c = n$, which yields the unique extremal construction of $G_1 = G_2$ being extremal graphs for $F$ on $n$ vertices.
\end{proof}

\section{Complete Graph $F$}

We will show the following result in this section:

\begin{theorem}\label{thm:induce-complete}
  For $n, m, r \geq 3$, 
  \[
    \dex^*(m, n, K_{r}) = m \cdot \ex(n, K_{r}),
  \]
  with equality for $n$-vertex graphs $G_1, G_2, \ldots, G_m$ if and only if $G_1 = \cdots = G_m = T_{r - 1}(n)$.
\end{theorem}

Surprisingly, the proof for the triangle case $(r = 3)$ is more complicated than the cases for larger $r$. Hence, we will first prove the case for $r \geq 4$ and then prove the triangle case separately. In particular, the case for $r \geq 4$ is a direct consequence of the following lemma:

\begin{lemma}\label{lem:induce-complete-cond}
  For $n \geq 2$ and $r \geq 3$,
  \[
    \ex(k, n_{r}) - \ex(n - 1, K_{r}) \geq \frac{n - 1}{2},
  \]
  with equality if and only if $n$ is odd and $r = 3$.
\end{lemma}

\begin{proof}
  By Turán's Theorem,
  \[
    \ex(k, n_{r}) - \ex(n - 1, K_{r}) = \delta(T_{r - 1}(n)) = n - \left\lceil \frac{n}{r - 1} \right\rceil \geq n - \left\lceil\frac{n}{2}\right\rceil.
  \]
  The result now follows.
\end{proof}

\begin{theorem}\label{thm:induce-complete-no-triangle}
  For $n, m \geq 2$ and $r \geq 4$, 
  \[
    \dex^*(m, n, K_{r}) = m \cdot \ex(k, K_{r}),
  \]
  with equality for $n$-vertex graphs $G_1, G_2, \ldots, G_m$ if and only if $G_1 = \cdots = G_m = T_{r - 1}(n)$.
\end{theorem}

\begin{proof}
  The result follows from \cref{lem:induce-cond} and \cref{lem:induce-complete-cond}.
\end{proof}

As shown in \cref{lem:induce-complete-cond}, the condition given by \cref{lem:induce-cond} is not satisfied for all $n$ in the triangle case, and there are indeed constructions of induced subgraphs $G_1, G_2$ that meet the extremal condition but are neither equal nor both complete bipartite graphs. For odd $n$, consider $G_1 = K_{\frac{n - 1}{2}, \frac{n - 1}{2}}$ and $G_2 = K_{\frac{n - 1}{2}, \frac{n - 1}{2}} + K_1$. The number of edges over $G_1, G_2$ is $\frac{(n - 1)^2}{2} + n - 1 = \frac{n^2 - 1}{2} = 2\left\lfloor \frac{n^2}{4}\right\rfloor$, which meets the benchmark construction of two complete bipartite graphs. We will show that this is the only deviant construction for the triangle case.

\begin{theorem}\label{thm:induce-triangle}
  Let $n, m \geq 2$, and let $G_1, \ldots, G_m$ be pairwise $K_3$-free induced subgraphs on $n$ vertices. Then
  \[
    \dex^*(m, n, K_3) = m \left\lfloor \frac{n^2}{4} \right\rfloor.
  \]
  Moreover, $\sum_i e(G_i) = \dex^*(m, n, K_3)$ if and only if $G_1 = \cdots = G_m = K_{\left\lfloor\frac{n}{2}\right\rfloor, \left\lceil\frac{n}{2}\right\rceil}$, unless $n$ is odd and $m = 2$, in which case $e(G_1) + e(G_2) = \dex^*(2, n, K_3)$ if and only if either $G_1 = G_2 = K_{\frac{n + 1}{2}, \frac{n - 1}{2}}$ or $G_{1} = K_{\frac{n - 1}{2}, \frac{n - 1}{2}}$ and $G_{2} = G_1 + K_1$.
\end{theorem}

\begin{proof}
  We first show the following claim.

  \begin{claim}\label{claim:induce-triangle}
    $\dex^*(2, n, K_3) = 2\left\lfloor\frac{n^2}{4}\right\rfloor$, and the extremal condition is met only when $G_1 = G_2 = K_{\left\lfloor\frac{n}{2}\right\rfloor, \left\lceil\frac{n}{2}\right\rceil}$, unless $n$ is odd, $G_1 = K_{\frac{n - 1}{2}, \frac{n - 1}{2}}$, and $G_2 = K_{\frac{n - 1}{2}, \frac{n - 1}{2}} + K_1$.
  \end{claim}

  \begin{proof}
    Consider $\Delta_2 \con(n, c, K_3)$. Since
    \[
      \Delta_2 \con(n, c, K_3) = -2c + 3 + 2\left[\left\lfloor\frac{c^2}{4}\right\rfloor - \left\lfloor\frac{(c - 2)^2}{4}\right\rfloor\right] = -2c + 3 + 2(c - 1) = 1 > 0,
    \]
    $\con(n, c, K_3)$ has a maximum of $2\left\lfloor \frac{n^2}{4} \right\rfloor$ when $c \geq n - 1$, so $\dex^*(2, n, K_3) = 2\left\lfloor\frac{n^2}{4}\right\rfloor$ by \cref{lem:optimize-con}. We are done if $c = n$, so assume that $c = n - 1$. Then in the extremal condition, $G_1 = G_{1, 2} = K_{\left\lfloor\frac{n - 1}{2}\right\rfloor, \left\lceil\frac{n - 1}{2}\right\rceil}$ and
    \[
      e(G_1) + e(G_2) = 2\left\lfloor\frac{(n - 1)^2}{4}\right\rfloor + \deg(v),
    \]
    where $v$ is the only vertex not in $G_{1, 2}$. But then to meet the extremal condition, 
    \[
      \deg(v) = 2\left\lfloor\frac{n^2}{4}\right\rfloor - 2\left\lfloor\frac{(n - 1)^2}{4}\right\rfloor = \begin{cases}
        n & \text{if $n$ is even}, \\
        n - 1 & \text{if $n$ is odd}.
      \end{cases}
    \]
    Hence, $n$ must be odd and $G_2$ must be a copy of $G_1$ with all vertices adjacent to the only remaining vertex, i.e. $G_2 = G_1 + K_1$.
  \end{proof}
  By \cref{lem:induce-reduce} and the above claim, it remains to show that for odd $n$ and $m = 3$, $G_1 = \cdots = G_3 = K_{\frac{n + 1}{2}, \frac{n - 1}{2}}$ if the extremal condition is met. Suppose not. The above claim then guarantees one of the subgraphs, say $G_1$, is $K_{\left\lceil\frac{n - 1}{2}\right\rceil, \left\lfloor\frac{n - 1}{2}\right\rfloor} + K_1$. But then by the claim $G_2 = G_3 = G_1 + K_1$, which contradicts that $G_2, G_3$ are pairwise $K_3$-free. This completes the proof.
\end{proof}

\cref{thm:induce-complete} now follows directly from \cref{thm:induce-complete-no-triangle} and \cref{thm:induce-triangle}. In fact, we proved that \cref{thm:induce-complete} also applies for $m = 2$, unless $r = 3$ and $n$ is odd.

% Since $n$ cannot be both even and odd, we also have the following corollary:

% \begin{corollary}\label{cor:induce-complete}
%   For $n \geq 2$, let $G_1, \ldots, G_n$ be pairwise $K_{r + 1}$-free induced subgraphs on $n$ vertices. Then
%   \[
%     \dex^*(n, n, K_{r + 1}) = n \cdot e(T_r(n)).
%   \]
%   and $\sum_i e(G_i) = \dex^*(n, n, K_{r + 1})$ if and only if $G_1 = \cdots = G_n = T_r(n)$.
% \end{corollary}

\section{Non-bipartite $F$}

For non-bipartite $F$, it is hard to determine the extremal graphs for $F$ in general, but their structures becomes more apparent when $n$ is large. 

More specifically, the Erdős-Stone Theorem tells us that for large $n$, the extremal graph for $F$ mimics the structure of the Turán graph. With this idea in mind, the following theorem is a generalization of \cref{thm:induce-complete} for large $n$.

\begin{theorem}
  For $m, r \geq 3$, $n$ large enough, and $r$-colorable graph $F$, 
  \[
    \dex^*(m, n, F) = m \cdot \ex(n, F),
  \]
  with equality for $n$-vertex graphs $G_1, G_2, \ldots, G_m$ if and only if $G_1 = \cdots = G_m$ are extremal $F$-free graphs.

  % and $\sum_i e(G_i) = \dex^*(m, n, F)$ if and only if $G_1 = G_2 = \cdots = G_n$ are $n$-vertex extremal graphs for $F$, unless $r = 2$, $n$ is odd, and $m = 2$, in which case $e(G_1) + e(G_2) = \dex^*(2, n, F)$ if and only if when either $G_1 = G_2$ are $n$-vertex extremal graph for $F$, or $G_{1}$ is the $(n - 1)$-vertex extremal graph for $F$ and $G_{2} = G_1 + K_1$.
\end{theorem}

\textcolor{red}{This proof only works for $r \geq 4$. Ignore the case $r = 3$ for now.}

\begin{proof}
  It suffices to show for $m = 2$ by \cref{lem:induce-reduce}. We first show that $\con(n, c, F)$ fails to meet the desired bound for small $c$.
  \begin{claim}
    If $c \leq \frac{n}{2}$, then $\con(n, c, F) < 2\ex(n, F)$.
  \end{claim}

  \begin{proof}
    Write $c = kn$ for some $k \in [0, 1/2]$. Since
    \[
      \con(n, kn, F) = \binom{(1 - k)n}{2} + k(1 - k)n^2 + 2\ex(kn, F),
    \]
    it suffices to show that
    \[
      \ex(n, F) - \ex(kn, F) > \frac{1}{2}\binom{(1 - k)n}{2} + \frac{k(1 - k)}{2}n^2
    \]
    for all $k \in [0, 1/2]$. By the Erdős-Stone theorem, $\ex(n, F) = \left(1 - \frac{1}{r - 1}\right)\frac{n^2}{2} + o(n^2)$ and so the left-hand-side is at least
    \[
      \ex(n, F) - \ex(kn, F) \geq \ex(n, F) - \ex(n/2, F) \geq \left(1 - \frac{1}{r - 1}\right)\left(\frac{n^2}{2} - \frac{n^2}{8}\right) - o(n^2) \geq \frac{3n^2}{16} - o(n^2).
    \]
    On the right-hand-side, 
    \[
      \frac{1}{2}\binom{(1 - k)n}{2} + \frac{k(1 - k)}{2}n^2 = \frac{1 - k^2}{4}n^2 + o(n^2) \leq \frac{n^2}{4} + o(n^2)
    \]
    \textcolor{red}{The problem is here. If $r = 3$, there does not exist $\alpha \in (0, 1]$ such that for $c \leq \alpha n$ the claim works: Erdős-Stone gives us $\ex(n, F) - \ex(kn, F) \geq \frac{1}{4}\left(1 - \alpha^2\right)n^2 + o(n^2)$, which exceeds the bound $\frac{n^2}{4} + o(n^2)$ for the right-hand-side for any $\alpha > 0$. }
  \end{proof}

  Thus we may assume that $c > \frac{n}{2}$. A theorem of Simonovits states that for large enough $c$, $\ex(c, F) = \ex(c, K_{r}) + \ex(c, \tilde{F})$, where $\tilde{F}$ is the family of residue subgraphs of $F$ after $F$ is embedded into $T_r(c)$. Since $\ex(c, \tilde{F})$ is non-decreasing on $c$,
  \[
    \ex(c, F) - \ex(c - 1, F) \geq \ex(c, K_{r}) - \ex(c - 1, K_{r}),
  \]
  as we assume $n$ is sufficiently large. Thus by \cref{lem:induce-cond} and \cref{lem:induce-complete-cond}, we are done for $r \geq 4$. 

  \textcolor{red}{The remaining proof is for $r = 3$.}
  
  The above inequality also implies that for $r = 3$,
  \[
    \Delta_2 \con(n, c, F) \geq \Delta_2 \con(n, c, K_3),
  \]
  which is positive by the proof of \cref{claim:induce-triangle}. Thus when $c$ is $n - 1$ or $n$, $\con(n, c, F)$ attains its maximum, and plugging in $c = n$ and $c = n -1$ yields 
  \[
    \con(n, c, F) \leq \max\left[2 \cdot \ex(n, F), 2 \cdot \ex(n - 1, F) + n - 1\right].
  \]
  By \cref{lem:induce-complete-cond},
  \[
    2 \cdot \ex(n, F) - [2 \cdot \ex(n - 1, F) + n - 1] \geq 2[\ex(k, n_{3}) - \ex(n - 1, K_{3})] - n + 1 \geq 0,
  \]
  with equality only if $n$ is odd. Hence, $\con(n, c, F) \leq 2 \cdot \ex(n, F)$. We may assume that $n$ is odd and $c = n - 1$, otherwise we are done by \cref{lem:induce-reduce}. Then in the extremal condition, $G_1 = G_{1, 2}$ is the extremal graph for $F$ on $n - 1$ vertices, and $G_2$ must be $G_1 + K_1$. It remains to show that for $m \geq 3$, $G_1 = \cdots = G_m$ are extremal graphs for $F$ when the extremal condition is met, and this follows from the argument in the proof of \cref{thm:induce-triangle}.
\end{proof}

% Similar to \cref{cor:induce-complete}, we have the following corollary:

% \textcolor{red}{This also only works for $r \geq 3$ because of the theorem above.}

% \begin{corollary}
%   Let $n , r \geq 2$, $F$ be a $(r + 1)$-colorable graph, and $G_1, \ldots, G_n$ be pairwise $F$-free induced subgraphs on $n$ vertices. Then for large enough $n$,
%   \[
%     \dex^*(n, n, F) = n \cdot \ex(n, F),
%   \]
%   and $\sum_i e(G_i) = \dex^*(n, n, F)$ if and only if $G_1 = G_2 = \cdots = G_n$ are extremal graphs for $F$.
% \end{corollary}

For small $n$, we may not be able to achieve the same result. Consider the case when $F$ is the bowtie graph, i.e. the $5$-vertex graph with two triangles sharing a vertex. For $n \leq 4$, the $n$-vertex extremal graph for $F$ is the complete graph $K_n$. For $n \geq 5$, the $n$-vertex extremal graph for $F$ is then $K_{\left\lfloor\frac{n}{2}\right\rfloor, \left\lceil\frac{n}{2}\right\rceil}$ plus an edge, and so $\ex(n, F) = \left\lfloor\frac{n^2}{4}\right\rfloor + 1$. But then in this case when $n = 5$, 
\[
  \con(5, 4, F) = 2e(K_4) + 4 = 16 > \con(5, 5, F) = 2\left(\left\lfloor\frac{5^2}{4}\right\rfloor + 1\right) = 14.
\]
This yields an instance where the construction $G_1 = K_{v(F) - 1}$ and $G_2 = K_n$ beats our benchmark construction. Thus the following lemma gives a lower bound for $n$ to avoid losing to this construction.

\begin{lemma}
  Let $n \geq 1$, $r \geq 2$, and $F$ be $(r + 1)$-colorable with $|V(F)| = t \geq 3$. If $n > t^2 - 3t + 2$ and $r$ divides $n$, then 
  \[
    \con(n, n, F) > \con(n, t - 1, F).
  \]
  \begin{proof}
    We need to show that
    \[
      2\ex(n, F) - \binom{n}{2} > \binom{t - 1}{2}.
    \]
    Since $\ex(n, F) \geq e(T_r(n)) = \left(1 - \frac{1}{r}\right)\frac{n^2}{2} \geq \frac{n^2}{4}$,
    \[
      2\ex(n, F) - \binom{n}{2} \geq \frac{n^2}{2} - \binom{n}{2} = \frac{n}{2} > \frac{t^2 - 3t + 2}{2} = \binom{t - 1}{2}.
    \]
  \end{proof}
\end{lemma}
\section{Bipartite $F$}

\section{Hypergraph $F$}

\chapter{General Version}

\textcolor{red}{TODO: add introduction.}

\begin{theorem}
  For all $n$ and graph $F$,
  \[
    \dex(m, n, F) = m(1 + o(1))\ex(n, F)
  \]
  as $m \to \infty$.
\end{theorem}

\begin{proof}
  Let $r = v(F)$. Pick $\epsilon > 0$. Reorder $G_1, \ldots, G_m$ so that $G_1, \ldots, G_{m'}$ are all the $G_i$'s containing at least $(1 + \epsilon)\text{ex}(n, F)$ edges. A theorem of Simonovits states that $G$ contains at least $\delta n^r$ copies of $F$ for some $\delta = \delta(\epsilon)$. Since there can be at most $\binom{n}{r}$ copies of $F$ across all $G_i$'s, 
  \[
    m'\delta n^{r} \leq \binom{n}{r} \leq n^r \implies m' \leq \frac{1}{\delta}.
  \]
  It now follows that
  \begin{align*}
    \sum_{i = 1}^m e(G_i) 
    &= \sum_{i = 1}^{m'} e(G_i) + \sum_{i = m' + 1}^m e(G_i) \\
    &\leq \frac{1}{\delta}\binom{n}{2} + \left(m - \frac{1}{\delta}\right)(1 + \epsilon)\text{ex}(n, F) \\
    &= m\left[1 + \epsilon + \frac{1}{m\delta}\left(\frac{\binom{n}{2}}{\ex(n, F)} - (1 + \epsilon)\right)\right]\ex(n, F).
  \end{align*}
  Since $\epsilon$ is arbitrary, the result follows.
\end{proof}

\begin{theorem}
  For large enough $n$, suppose that $G_1, \ldots, G_m$ are graphs on common vertex set $[n]$ with no copy of $F$ contained in any $k$ of the $G_i$'s. If there exists extremal $F$-free subgraph $H$ on $n$ vertices such that $\binom{m}{k}\Delta(H) = o(n^{1/2})$, then
  \[
    \dex(m, n ,F) = (k - 1)\binom{n}{2} + \ex(n, F)\binom{m}{k}.
  \]
\end{theorem}

\begin{proof}
  For $S \subseteq [m]$, let $E_S$ denote the set of edges that are contained in exactly $\{G_i\}_{i \in S}$. Then 
  \[
    \sum_{i = 1}^m e(G_i) = \sum_{S \subseteq [m]} |S||E_S| \leq (k - 1)\binom{n}{2} + \sum_{S \subseteq [m], |S| \geq k} (|S| - k + 1)|E_S|.
  \]
  Let $A_S = \bigcup_{T \supseteq S} E_T$, i.e. the set of edges that are contained in all $G_i$ with $i \in S$. When $|S| \geq k$, the edge set $A_S$ is $F$-free and thus 
  \[
    \sum_{T \supseteq S} |E_T| \leq \text{ex}(n, F).
  \]
  Hence,
  \[
    \sum_{\substack{S \subseteq [m] \\ |S| \geq k}} (|S| - k + 1)|E_S| = \sum_{\substack{S \subseteq [m], \\ |S| = k}} \sum_{T \subseteq S} \frac{(|T| - k + 1)|E_T|}{\binom{|T|}{k}} \leq \sum_{\substack{S \subseteq [m], \\ |S| = k}} \sum_{T \subseteq S} |E_T| \leq \binom{m}{k}\text{ex}(n, F),
  \]
  as each $T \in [m]$ with $|T| \geq k$ is counted $\binom{|T|}{k}$ times in total and $|T| - k + 1 \leq \binom{|T|}{k}$. This proves the upper bound.

  Now we show the bound is tight. In particular, we need to show there exists a construction such that the graph with edge set $E_S$ is an extremal $F$-free graph, for all $S \subseteq [m]$ of size $k$. Let $M = \binom{m}{k}$ and $H_1, \ldots, H_M$ be copies of an extremal $F$-free graph on $n$ vertices with $\Delta(H_i) = o(n^{1/2})$ for all $i$. It suffices to show that we can embed each $H_i$ onto $[n]$ such that their edge sets are pairwise disjoint. We begin by an arbitrary embedding of each $H_i$ and iteratively decrease the number of intersecting edges. Define a $(u, v, i)$-swap by swapping the embedding of vertex $u$ and $v$ of $H_i$, i.e. replacing each edge $\{u, w\} \in E(H_i)$ with the edge $\{u, w\}$ and each edge $\{v, w\} \in E(H_i)$ with the edge $\{v, w\}$. This perserves the type of isomorphism of $H_i$. Given a vertex $v$, let $N(v) = N_{H_1}(v) \cup \cdots \cup N_{H_M}(v)$. Suppose there exists an intersecting edge $\{u, w\} \in E(H_i) \cap E(H_j)$. Since $|N(u)|
  \leq M \cdot \Delta(H_i) = o(n^{1/2})$, $|N(u) \cup N(N(u))| = o(n)$ so there exists a vertex $v \notin N(u) \cup N(N(u))$. Since $N(u) \cap N(v) = \emptyset$, performing a $(u, v, i)$-swap reduces the number of intersecting edges. The result now follows from iterating this process.
\end{proof}


\section{Complete $F$}

It turns out that $\dex(m, n, K_r)$ can be determined exactly, but the statement of the theorem requires some definitions. Let $k \geq 2$ and sets $S_{ij} \subseteq [m]$, for $1 \leq i < j \leq k$. We call the following type of construction a \textit{$k$-blowup}: 

Define $G_1, G_2, \ldots, G_m$ by partitioning $[n]$ into $k$ sets $V_1, V_2, \ldots, V_k$ and letting $\{u, v\} \in E(G_h)$ whenever $u \in V_i, v \in V_j, h \in S_{ij}$. Additionally, for each $i \in [k]$ and $\{u, v\} \subseteq V_i$, we place $\{u, v\}$ in exactly one of $G_1$. 

Given a $k$-blowup, we may define $m$ graphs $H_1, H_2, \ldots, H_m$ on $[k]$ with edge set $E(H_h) \coloneq \{\{i, j\} : h \in S_{ij}\}$ for $h \in [m]$. We call $H_1, H_2, \ldots, H_m$ the \textit{pattern} of the $k$-blowup, and we say that a $k$-blowup is doubly $F$-free if its pattern is doubly $F$-free. 

\begin{center}
  \begin{tikzpicture}
    \draw (4, 2) to node[midway, above] {$123$} (-4, 2); \draw (-4, 2) to node[midway, left] {$124$} (-1, -1); \draw (-1, -1) to node[midway, above] {$34$} (1, -1); \draw (1, -1) to node[midway, right] {$124$} (4, 2); \draw (0, 1) to node[midway, below] {$234$} (4, 2); \draw (0, 1) to node[midway, below] {$134$} (-4, 2); \draw (0, 1) to node[midway, left] {$123$} (-1, -1); \draw (0, 1) to node[midway, right] {$123$} (1, -1); \draw[bend left=80] (4, 2) to node[midway, right] {$134$} (-1, -1); \draw[bend right=80] (-4, 2) to node[midway, left] {$234$} (1, -1);

    \foreach \x/\y in {0/1, 4/2, 1/-1, -1/-1, -4/2} { \fill (\x, \y) circle (2pt); }
  \end{tikzpicture}
  \\
  \small{Example of a 5-blowup. A label $i \in \{1, 2, 3, 4\}$ at an edge indicates that the edge is in $G_i$.}
\end{center}

For $M \geq 3$, let $R_M(F)$ denote the $M$-color Ramsey number for $F$. That is, the minimum number $N$ such that there exists a monochromatic $F$ in any $M$-coloring of the edges of $K_N$.

Let $m \geq 3$ and $M = \binom{m}{2}$. Define $B(m, n, F)$ as the maximum of $\sum_{i = 1}^m e(G_i)$ such that $G_1, G_2, \ldots, G_m$ form a doubly $F$-free $k$-blowup, for some $k \leq R_M(F)$.

We are new ready to state the theorem.

\begin{theorem}\label{thm:complete-blowup}
  For $n, m \geq 1$ and $r \geq 2$, 
  \[
    \dex(m, n, K_r) = B(m, n, K_r).
  \]
\end{theorem}

\begin{proof}[Proof of \cref{thm:complete-blowup}]
  Notice that we trivially have $B(m, n, K_r) \leq \dex(m, n, K_r)$, so it suffices to show the reverse inequality. That is, we need to show that there exist $k \leq R_{M}(K_r)$ and $S_{ij}$ for $1 \leq i < j \leq k$ such that the $k$-blowup construction on meets the desired bound.

  Let $G_1, G_2, \ldots, G_m$ be graphs on $[n]$ with no double $K_r$ and $\sum_{i = 1}^m e(G_i) = \dex(m, n, K_r)$. Observe that any pair $\{i, j\} \subseteq [n]$ must be in some $G_i$, otherwise we may add it to $G_1$ without creating a double $K_r$. 

  We call vertices $v, v'$ \textit{clones} if for all $u \in [n] \backslash \{v, v'\}$ and $i \in [m]$, the edge $\{u, v\} \in E(G_i)$ if and only if $\{u, v'\} \in E(G_i)$. Furthermore, we call $\{v, v'\}$ a \textit{light edge} if $\{v, v'\}$ is in exactly one graph $G_i$.

  We now apply \cref{alg:sym} to $G_1, G_2, \ldots, G_m$. 

  \begin{algorithm}[H]
    \caption{symmetrization algorithm}\label{alg:sym}
    \begin{algorithmic}
      \While{$\exists$ a light edge whose endpoints are not clones}
      \State{among all vertices incident to such an edge, select a vertex $v$ with maximum degree}\;
      \State{$B_v\gets$ collection of vertices sending a light edge to $v$ that are not clones of $v$}\;
      \While{$B_v\neq \emptyset$}
      \State{pick $u\in B_v$}\;
      \State{$j \gets$ colour of the light edge from $u$ to $v$}\;
      \For{$1\leq i\leq m$}\;
      \If{$i\neq j$};
      \State{$N_{G_i}(u) \gets N_{G_i}(v)$}\;
      \ElsIf{$i=j$}
          \State{$N_{G_i}(u) \gets \left(N_{G_i}(v)\setminus \{u\}\right)\cup\{v\}$}\;
      \EndIf		
      \EndFor
      \EndWhile
      \EndWhile
    \end{algorithmic}
  \end{algorithm}

  \begin{claim}
    \cref{alg:sym} terminates.
  \end{claim}

  \begin{proof}
    Notice that at the end of the `while $B_v \neq \emptyset$' loop, every vertex sending a light edge to $v$ is a clone of $v$. This implies $v$ along with the set $L_v$ of vertices receiving light edges from $v$ induce a clique of size at least two in some $G_i$, and an empty graph in every other graph $G_j$ with $j \neq i$. Moreover, any vertex $w \notin L_v$ sends edges to either all or none of the vertices in $L_v$, and if $w$ is incident to $L_v$, then $w$ sends edges to $L_v$ in at least two graphs. It now follows that no light edge incident with a vertex in $L_v$ will be picked again in an iteration of the out most while loop. Thus the algorithm can run through at most $n/2$ such iterations, and so it terminates.
  \end{proof}

  \begin{claim}
    $G_1', G_2', \ldots, G_m'$ do not contain a double $K_r$ and $\sum_{i = 1}^m e(G_i') = \dex(m, n, K_r)$. 
  \end{claim}

  \begin{proof}
    Note that we replace $u$ by a clone of $v$ in the for loop of \cref{alg:sym}. Since $\{u, v\}$ remains to be an light edge in this step, $u$ and $v$ cannot both belong to a double $K_r$ in the modified graphs. Furthermore, any double $K_r$ containing $u$ after the for loop arises from a double $K_r$ containing $v$ prior to the for loop. But then $G_1, G_2, \ldots, G_m$ contained no double $K_r$ to begin with, so $G_1', G_2', \ldots, G_m'$ do not contain a double $K_r$.

    We now show that the algorithm does not reduce the number of edges. By our choice of $v$, we know $d(v) \geq d(u)$ for all $u \in B_v$ prior to the for loop. Hence, replacing $u$ with a clone of $v$ does not decrease the number of edge over a complete iteration of the inner while loop. Therefore, $\sum_{i = 1}^m e(G_i') = \dex(m, n, K_r)$. 
  \end{proof}

  Hence, the algorithm outputs graphs $G_1', G_2', \ldots, G_m'$ with $\dex(m, n, K_r)$ edges and the additional property that light edges come in `clone cliques.' We may thus partition the vertex set $[n]$ into $k$ disjoint sets $V_1, V_2, \ldots, V_k$, such that each $V_i$ induces a clique of light edges from the same graph. Moreover, for distinct $i, j \in [k]$, define $S_{ij}$ to be the set of all edges between $V_i$ and $V_j$, and note that any edge in $S_{ij}$ appear in at least two modified graphs. The sets $S_{ij}$ now yield a $k$-blowup. Notice that if the pattern of the $k$-blowup contains a double $K_r$, then the original graphs $G_1, G_2, \ldots, G_m$ must have contained a double $K_r$ as well, contradiction. Thus the $k$-blowup is doubly $K_r$-free. 

  It remains to show that $k < R_M(K_r)$. For each edge $\{i, j\} \subseteq [k]$ in the pattern of the $k$-blowup, we assign an arbitrary distinct pair $\{a, b\} \subseteq L_{ij} \subseteq [m]$ to $\{i, j\}$. If $k \geq R_M(K_r)$, then there exists $K_r$ in the pattern of the $k$-blowup colored by some distinct pair $\{a, b\} \subseteq [m]$. But then this implies the pattern of the $k$-blowup contains a double $K_r$, contradiction. This completes the proof.
\end{proof}

% Consider $F$ to be a triangle. Simply counting the number of triangles in each $G_i$ shows the following:

% \begin{theorem}
%   For all $n, m$ and $\epsilon > 0$,
%   \[
%     \dex(m, n, K_3) < \left(m \cdot \frac{1 + \epsilon}{4} + \frac{1}{2\epsilon} - \frac{1}{2}\right)n^2.
%   \]
% \end{theorem}

% \begin{proof}
%   We first show that there are less than $\frac{2}{\epsilon}$ number of $G_i$'s with at least $(1 + \epsilon)\frac{n^2}{4}$ edges. Suppose $e(G_i) \geq (1 + \epsilon)\frac{n^2}{4}$ for $1\leq i \leq k$. Let $K_3(G)$ denote the number of triangles in graph $G$. By the Moon-Moser inequality,
% 	\[
% 		K_3(G_i) \geq \epsilon(1 + \epsilon)\frac{n^3}{12}.
% 	\]
% 	Since there are no overlapping traingles from different $G_i$'s, 
% 	\[
% 		\binom{n}{3} \geq \sum_{i = 1}^k K_3(G_i) \geq \frac{\epsilon(1 + \epsilon)}{12}kn^3.
% 	\]
% 	Rearranging yields $k < \frac{2}{\epsilon}$.

%   It now follows that
%   \[
%     \sum_{i = 1}^m e(G_i) < \frac{2}{\epsilon}\binom{n}{2} + \left(m - \frac{2}{\epsilon}\right)(1 + \epsilon)\frac{n^2}{4} \leq m(1 + \epsilon)\frac{n^2}{4} + (1 - \epsilon)\frac{n^2}{2\epsilon},
%   \]
%   and this completes the proof.
% \end{proof}

When $m = 2$, 
\[
	e(G_1) + e(G_2) \leq \binom{n}{2} + e(G_{1, 2}) \leq \binom{n}{2} + \ex(n, K_3)
\]
which meets the benchmark bound. Surprisingly, our desired bound is also met when $m = 3$:

\begin{theorem}
  For all $n$,
  \[
    \dex(3, n, K_3) = \binom{n}{2} + \left\lfloor \frac{n^2}{2} \right\rfloor.
  \]
\end{theorem}

\begin{proof}
  Define $H_k \subseteq G$ be the graph with edges contained in at least $k$ number of $G_i$'s and note that $e(G_1) + e(G_2) + e(G_3) = e(H_1) + e(H_2) + e(H_3)$. Thus it suffices to show that $e(H_2) + e(H_3) \leq \frac{n^2}{2}$. Notice $H_2$ must not contain any triangles with two edges in $H_3$, so
  \[
    e(H_2) + e(H_3) \leq \binom{n}{2} + e(H_3) - |\{\{u, v\} : u \neq v, N_{H_3}(u) \cap N_{H_3}(v) \neq \emptyset\}|.
  \]
  Let $H'_3$ be the graph with the same vertex set as $H_3$ and edge set $\{\{u, v\} : u \neq v, N_{H_3}(u) \cap N_{H_3}(v) \neq \emptyset\}$. It suffices to show that $\frac{n}{2} \geq e(H_3) - e(H'_3)$. 

  Let $d_1 \geq d_2 \geq \cdots \geq d_n$ and $f_1 \geq f_2 \geq \cdots \geq f_n$ each be the degree sequence of $H_3$ and $H_3'$, respectively. We show that $f_i \geq d_i - 1$ for all $i$. Let $v_i$ denote the vertex in $H$ with degree $d_i$ and $u_i$ be the vertex in $H$ with degree $f_i$. Let $S_i = |N_{H_3}(v_1) \cup \cdots \cup N_{H_3}(v_i)|$. Since 
  \[
    \sum_{u \in S_i} d_{H_3}(u) \geq d_1 + \cdots + d_i,
  \]
  we have that $|S_i| \geq i$. But then $S_i \backslash \{u_1, \ldots, u_{i - 1}\}$ is non-empty, and every $u \in S_i$ has degree $d_{H_3'}(u) \geq d_i - 1$. Hence, $f_i \geq d_i - 1$ for all $i$, which yields
  \[
    e(H_3') = \frac{1}{2}\sum_{i = 1}^n f_i \geq \frac{1}{2}\sum_{i = 1}^n (d_i - 1) = e(H_3) - \frac{n}{2}.
  \]
\end{proof}

However, the bound in Proposition 3.1 is not tight for $m \geq 4$, as shown in the following graph:

\begin{center}
  \begin{tikzpicture}
    \draw (4, 2) to node[midway, above] {$123$} (-4, 2); \draw (-4, 2) to node[midway, left] {$124$} (-1, -1); \draw (-1, -1) to node[midway, above] {$34$} (1, -1); \draw (1, -1) to node[midway, right] {$124$} (4, 2); \draw (0, 1) to node[midway, below] {$234$} (4, 2); \draw (0, 1) to node[midway, below] {$134$} (-4, 2); \draw (0, 1) to node[midway, left] {$123$} (-1, -1); \draw (0, 1) to node[midway, right] {$123$} (1, -1); \draw[bend left=80] (4, 2) to node[midway, right] {$134$} (-1, -1); \draw[bend right=80] (-4, 2) to node[midway, left] {$234$} (1, -1);

    \foreach \x/\y in {0/1, 4/2, 1/-1, -1/-1, -4/2} { \fill (\x, \y) circle (2pt); }
  \end{tikzpicture}
  \\
  \small{The number on each edge denotes the set of $G_i$'s that contain the edge.}
\end{center}
The above graph contains $29$ edges, which exceeds the bound $\binom{5}{2} + 3 \lfloor \frac{5^2}{4} \rfloor = 28$ by $1$. By blowing up the above graph, we can construct a graph with $n \in 10\mathbb{Z}$ vertices that contains
\[
  5\binom{n/5}{2} + 29 \cdot \frac{(n/5)^2}{4}
\]
edges, which exceeds the bound $\binom{n}{2} + 3\lfloor\frac{n^2}{4}\rfloor$ by $n^2/100$. 



\section{Bipartite $F$}

In this section, we discuss the case where $F$ is bipartite. In particular, we focus on the cases where $F \subseteq K_{2, 2}$ is $P_2$, a path of length $2$, or $M_2$, a matching with two edges.

\begin{theorem}
  \[
    \dex(m, n, P_2) \leq \left(\frac{1}{2} + o(1)\right)\min\{n^{2}\sqrt{m}, mn^{3/2}\},
  \]
  as $n \to \infty$ or $m \to \infty$. Moreover, 
  \[
    \dex(m, n, P_2) = \left(\frac{1}{2} + o(1)\right)mn^{3/2},
  \]
  for $\sqrt{n} \leq m \leq n$.
\end{theorem}

\begin{proof}
  Let $G_1, \ldots, G_m$ be graphs on $[n]$ not containing a $P_2$. We first show the claimed upperbound and then show the tightness of the bound when $\sqrt{n} \leq m \leq n$.
  \begin{claim}
    $\dex(m, n, P_2) \leq mn \cdot \frac{1 + \sqrt{4n^2/m + 1}}{4}$.
  \end{claim}

  \begin{proof}
    Since there are no double $P_2$,
    \[
    \sum_{i = 1}^m \#\{P_2 \subseteq G_i\} \leq \#\{P_2 \subseteq G\}.
    \]
    For all $G_i$, each vertex $v$ in $G_i$ along with two of its neighbors form one unique $P_2$, so
    \[
      \#\{P_2 \subseteq G_i\} = \sum_{v \in V(G_i)} \binom{d_{G_i}(v)}{2}.
    \]
    By Jensen's inequality,
    \[
      \sum_{v \in V(G_i)} \binom{d_{G_i}(v)}{2} \geq n\binom{d_{G_i}(v)/n}{2} = n\binom{2e(G_i)/n}{2} \geq \frac{2(e(G_i))^2}{n} - e(G_i).
    \]
    On the other hand, since each three vertices in $G$ can form at most three $P_2$'s, 
    \[
      \#\{P_2 \subseteq G\} \leq 3\binom{n}{3} \leq \frac{n^3}{2}.
    \]
    Combining the above inequalities yields and using Jensen's inequality once more yields
    \[
      \frac{2m}{n}\left(\frac{1}{m}\sum_{i = 1}^m e(G_i)\right)^2 - \sum_{i = 1}^m e(G_i) \overset{Jensen's}{\leq} \sum_{i = 1}^m \frac{2(e(G_i))^2}{n} - e(G_i) \leq \frac{n^3}{2}.
    \]
    Solving the quadratic equation gives
    \[
      \sum_{i = 1}^m e(G_i) \leq mn \cdot \frac{1 + \sqrt{4n^2/m + 1}}{4}.
    \]
  \end{proof}

  \begin{claim}
    $\dex(m, n, P_2) \leq \frac{1}{2}(mn^{3/2} + n^2)$. 
  \end{claim}

  \begin{proof}
    For each vertex $u \in [n]$, define $H_u$ as the $m \times n$ bipartite graph with edge set $E(H_u) \coloneq \{\{v, i\} : \{u, v\} \in E(G_i)\}$. If $H_u$ contains a quadrilateral $\{v, i\}, \{v, j\}, \{w, i\}, \{w, j\}$, then $\{u, v\}, \{u, w\}$ form a double $P_2$ in $G_i \cap G_j$, contradiction. Thus we conclude that $H_u$ is quadrilateral-free, and therefore $e(H_u) \leq m\sqrt{n} + n$, by the Kővari-Sós-Turán theorem. It now follows that
    \[
      \sum_{i = 1}^m e(G_i) = \frac{1}{2}\sum_{u \in V(G)} e(H_u) \leq \frac{1}{2}(mn^{3/2} + n^2).
    \]
  \end{proof}
  The above two claims yield the desired upper bound. We now show the lower bound.
  \begin{claim}
    $\dex(m, n, P_2) \geq (1/2 + o(1))mn^{3/2}$, for $\sqrt{n} \leq m \leq n$.
  \end{claim}
  \begin{proof}
    Suppose $G_1, G_2, \ldots, G_n$ are graphs on $[n]$ containing no double $P_2$ and $\sum_{i = 1}^n e(G_i) \geq (1/2 + o(1))n^{5/2}$, with $e(G_1) \geq e(G_2) \geq \cdots \geq e(G_n)$. Then $G_1, G_2, \ldots, G_m$ are graphs with no double $P_2$ and $\sum_{i = 1}^m e(G_i) \geq (1/2 + o(1))mn^{3/2}$. Hence, it suffices to prove the case for $m = n$.

    Consider a finite projective plane of order $q$. The projective plane has $n = q^2 + q + 1$ points and $n$ lines, where $q$ is a prime chosen so that $n = (1 + o(1))(q^2 + q + 1)$ as $q \to \infty$. Let $S_1, \ldots, S_n \subseteq [n]$ be the $n$ lines of the projective plane. Note that each line $S_i$ contains $q + 1$ points, and the intersection of any two distinct lines $S_i, S_j$ contains $|S_i \cap S_j| = 1$ point. 
    
    Define $G_1, \ldots, G_n$ to be graphs on $[n]$, each with edge set
    \[
      E(G_i) \coloneq \{\{j, k\} \subseteq [n] : j \neq k, \, j + k \in S_i \mod n\}.
    \]
    Note that the intersection of distinct $G_i$, $G_j$ is $P_2$ free: since $|S_i \cap S_j| = 1$, if $\{a, b\}, \{a, c\} \in E(G_i) \cap E(G_j)$, then $a + b = a + c$ so $b = c$. 
    
    We now count the number of edges in $G_1, \ldots, G_n$. Since $|S_i| = q + 1$, for each point $j \in [n]$, there are $q + 1$ choices for $k \in [n]$ such that $j + k \in S_i$. But then we have to avoid counting the same edge twice and loops, so the number of edges in $G_i$ is
    \[
      e(G_i) = \frac{n(q + 1) - \#\text{loops counted for } G_i}{2}.
    \]
    If $j \in [n]$ is even, then $k = j/2$ is the unique number in $[n]$ such that $k + k = j \mod n$. If $j \in [n]$ is odd, then $k = (n + j)/2$ is the unique number in $[n]$ such that $k + k = j \mod n$, as $n$ is even. Hence, for each $j \in S_i$, there exists a unique $k \in [n]$ such that $k + k = j \mod n$, and thus
    \[
      \#\text{loops counted for } G_i = |S_i| = q + 1.
    \]
    Since $q + 1 = (1 + o(1))n^{1/2}$, the number of edges in $G_1, \ldots, G_n$ is
    \[
      \sum_{i = 1}^n e(G_i) = n \cdot \frac{n(q + 1) - (q + 1)}{2} = \left(\frac{1}{2} + o(1)\right)n^{5/2},
    \]
    as $n \to \infty$.
  \end{proof}
  \begin{claim}
    $\dex(m, n, P_2) \geq (1/2 + o(1))\sqrt{m}n^{2}$, for $n < m \leq n^2$.
  \end{claim}
  \begin{proof}
    \textcolor{red}{I don't understand the proof for this claim.}
  \end{proof}
\end{proof}

\begin{theorem}
  For all $n, m$,
  \[
    \dex(m, n, M_2) \leq n^{5/2}.
  \]
\end{theorem}

\begin{proof}
  Notice that $\#\{M_2 \subseteq G\} = \binom{e(G_i)}{2}$. On the other hand, each four vertices in $G$ can form at most three $M_2$'s, so $\#\{M_2 \subseteq G\} \leq 3\binom{n}{4} \leq \frac{n^4}{8}$. By the same argument as in Theorem 3.4, we have 
  \[
    \sum_{i = 1}^n \binom{e(G_i)}{2} = \sum_{i = 1}^n \#\{M_2 \subseteq G_i\} \leq \#\{M_2 \subseteq G\} \leq \frac{n^4}{8}.
  \]
  By Jensen's inequality,
  \[
    \sum_{i = 1}^n \binom{e(G_i)}{2} \geq n\binom{\sum_{i = 1}^n e(G_i)/n}{2} = \frac{1}{2n}\left[\left(\sum_{i = 1}^n e(G_i)\right)^2 - n\sum_{i = 1}^n e(G_i)\right].
  \]
  Combining the above inequalities yields
  \[
    \left(\sum_{i = 1}^n e(G_i)\right)^2 - n\sum_{i = 1}^n e(G_i) \leq \frac{n^5}{4},
  \]
  and solving the quadratic inequality gives
  \[
    \sum_{i = 1}^n e(G_i) \leq n^{5/2}.
  \]
\end{proof}

We may obtain an exact result if we forbid both $P_2$ and $M_2$ at the same time:

\begin{theorem}
  For all $n, m$,
  \[
    \dex(m, n, \{P_2, M_2\}) = n^2 - n.
  \]
\end{theorem}

\begin{proof}
  Denote the set of $G_i$'s as $\{G_i\} = \{G_1, \ldots, G_n\}$, and the set of distinct pairs of $G_i$'s as $\{G_i\}^2 = \{\{G_j, G_k\} : j \neq k\}$. Consider the bipartite graph $H$ with vertex set $V(H) = \{G_i\} \sqcup E(K_n)$ and edge set $E(H) = \{\{G_j, e\} \in \{G_i\} \times E(K_n) : e \in G_j\}$. Define $\phi: \{G_i\}^2 \to 2^{E(K_n)}$ by sending each $\{G_j, G_k\}$ to their common edge set $E(G_j) \cap E(G_k)$. Notice that each distinct $G_j, G_k$ have at most one edge in common, so $|\phi(G_j, G_k)| \leq 1$. On the other hand, each edge $e \in E(G)$ can be obtained via $\phi$ by $\binom{d_H(e)}{2}$ possible distinct pairs $(G_j, G_k)$, and thus $|\phi^{-1}(e)| = \binom{d_H(e)}{2}$. But then
  \[
    \binom{n}{2} \geq \sum_{(G_j, G_k) \in \{G_i\}^2} |\phi(G_j, G_k)| = \sum_{e \in E(K_n)} |\phi^{-1}(e)| = \sum_{e \in E(K_n)} \binom{d_H(e)}{2}.
  \]
  By Jensen's inequality,
  \[
    \sum_{e \in E(K_n)} \binom{d_H(e)}{2} \geq \binom{n}{2}\binom{\sum_{e \in E(K_n)} d_H(e)/\binom{n}{2}}{2} = \binom{n}{2}\binom{\sum_{i = 1}^n e(G_i)/\binom{n}{2}}{2}.
  \]
  Combining the above inequalities yields
  \[
    2\binom{n}{2}^2 \geq \left(\sum_{i = 1}^n e(G_i)\right)^2 - \binom{n}{2}\sum_{i = 1}^n e(G_i),
  \]
  and the result now follows from solving the quadratic inequality.

  To see that this bound is tight, consider the construction such that for each distinct $i, j \in [n]$, $E(G_i) \cap E(G_j)$ contains exactly one unique edge $e \in K_n$. The number of edges in this construction is $2\binom{n}{2} = n^2 - n$.
\end{proof}

\end{document}
