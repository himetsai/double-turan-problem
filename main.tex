\documentclass[12pt]{report}
\usepackage{amsmath, amssymb, amsthm}
\usepackage{geometry}
\usepackage{graphicx}
\usepackage{hyperref}
\usepackage{listings}
\usepackage{enumitem}
\usepackage{titlesec}
\usepackage{setspace}
\usepackage{mathtools}
\usepackage{tikz}
\usepackage{tikz-cd}
\usepackage[colorinlistoftodos]{todonotes}
\titleformat{\chapter}[hang]{\normalfont\huge\bfseries}{\thechapter}{1em}{}
\usetikzlibrary{automata,positioning}

% Page layout
\geometry{a4paper, margin=1in}
\setlength{\parskip}{5pt}

% Theorem Environments
\newtheorem{theorem}{Theorem}[chapter]
\newtheorem{lemma}{Lemma}[theorem]
\newtheorem{claim}{Claim}
\newtheorem{conjecture}{Conjecture}[chapter]
\newtheorem{definition}[theorem]{Definition}
\newtheorem{corollary}[theorem]{Corollary}
\newtheorem{proposition}[theorem]{Proposition}

% Title Page
\title{Double Turán Problem}
\author{Ray Tsai}
\date{November 2024}

\begin{document}

\maketitle

% \begin{abstract}
% Your abstract here. Provide a concise summary of your thesis, including the problem statement, methodology, main results, and conclusions.
% \end{abstract}

% \chapter*{Acknowledgments}
% Thank your advisor, collaborators, and anyone who supported your research.

\tableofcontents

\chapter{Introduction}

\textcolor{red}{TODO: add more introduction.}

\section{Definitions and Notation}
Let $G = (V, E)$ be a graph. Let $V(G) = V$ denote the vertex set and $E(G) = E$ denote the edge set
of $G$. We note by $v(G) = |V|$ the number of vertices and $e(G) = |E|$ the number of edges in $G$.
For vertex $v \in V(G)$, we denote by $N_G(v) = \{u \in V(G) : \{u, v\} \in E(G)\}$ the neighborhood
of $v$.

Given $G_1, \ldots, G_m$ subgraphs of $G$, we denote $G_{i_1, \ldots, i_k}$ as the subgraph of $G$
with edge set $E(G_{i_1, \ldots, i_k}) = \bigcap_{\alpha = 1}^k E(G_{\alpha})$. 

In this thesis, we reserve $n$ to denote the number of vertices in a graph. Given a graph $F$, we
denote $\text{ex}(n, F)$ to be the extremal number for $F$ on a graph with $n$ vertices, i.e. the
maximum number of edges in a $n$-vertex graph that does not contain $F$ as a subgraph.

We call a $n$-vertex complete graph $K_n$, and a complete bipartite graph $K_{a, b}$, where $a, b$
are the size of its parts. We denote $P_n$ as a path with $n$ edges, and $C_n$ as a cycle with
$n$ edges. Given graph $G, H$, define $G + H$ as the graph fully connecting $G, H$, i.e. $V(G + H) =
V(G) \cup V(H)$ and $E(G + H) = E(G) \cup E(H) \cup \{\{u, v\} : u \in V(G), v \in V(H)\}$.

We also denote the set of first $n$ positive integers as $[n] = \{1, 2, \ldots, n\}$. Given a set
$X$, we denote $2^X$ as the power set of $X$.

\section{Problem Statement}

Given graph $G$ with $n$ vertices, let $G_1, \ldots, G_m$ be subgraphs of $G$. Let $F$ be a graph
with at least one edge. Our goal is to determine the maximum sum of the number of edges over all
$G_i$'s, i.e. $\sum_{i = 1}^m e(G_i)$, with the constraint of $E(G_i) \cap E(G_j)$ not including
some graph $F$ for all distinct $i, j$. 

In this thesis, we first consider the case where $G_1, \ldots, G_m$ are induced subgraphs, and then
shift our focus to the general case. At the end, we will discuss the case where $F$ is bipartite.

\chapter{Induced Case}

In this section, we assume that $G_1, \ldots, G_m$ are induced subgraphs of $G$. We first show a
simpler case where $F$ is a triangle.

\section{Triangle $F$}

\begin{theorem}
  Suppose that $E(G_i) \cap E(G_j)$ does not include $K_3$ for distinct $i, j$. Then
  \[
    \sum_{i = 1}^n e(G_i) \leq n\left\lfloor\frac{n^2}{4}\right\rfloor,
  \]
  with equality if and only if $G_1 = G_2 = \cdots = G_n = K_{\left\lceil\frac{n}{2}\right\rceil,
  \left\lfloor\frac{n}{2}\right\rfloor}$.
\end{theorem}

First we prove the case for two graphs:

\begin{lemma}
  Suppose $E(G_1) \cap E(G_2)$ does not include $K_3$. Then
  \[
    e(G_1) + e(G_2) \leq 2\left\lfloor\frac{n^2}{4}\right\rfloor,
  \]
  with equality if and only if $G_1 = G_2 = K_{\left\lceil\frac{n}{2}\right\rceil,
  \left\lfloor\frac{n}{2}\right\rfloor}$, unless $n$ is odd and $G_1 = K_{\left\lceil\frac{n -
  1}{2}\right\rceil, \left\lfloor\frac{n - 1}{2}\right\rfloor}$ and $G_2 = K_{\left\lceil\frac{n -
  1}{2}\right\rceil, \left\lfloor\frac{n - 1}{2}\right\rfloor} + K_1$.
\end{lemma}

\begin{proof}
  Let $C = V(G_1) \cap V(G_2)$, $A = V(G_1) \backslash C$, and $B = V(G_2) \backslash C$. Put $a =
  |A|$, $b = |B|$, and $c = |C|$. We may assume that $a + b + c = n$.

  We find an upper bound for $e(G_1) + e(G_2)$ with respect to $a, b, c$. Since $G_1, G_2$ are
  induced graphs, $\{u, v\} \in E(G_1)$ if and only if $\{u, v\} \in E(G_2)$, for $u, v \in C$, and
  thus $E(G_1[C]) = E(G_2[C]) = E(G_i) \cap E(G_j)$. But then $E(G_i) \cap E(G_j)$ is triangle-free,
  so $e(G_1[C]) \leq \left\lfloor\frac{c^2}{4}\right\rfloor$, with equality if and only if $G_1[C] =
  K_{\left\lceil\frac{c}{2}\right\rceil, \left\lfloor\frac{c}{2}\right\rfloor}$. Hence, 
  \begin{gather}
    e(G_1) + e(G_2) \leq \binom{a + c}{2} + \binom{b + c}{2} - 2\left[\binom{c}{2} - \left\lfloor\frac{c^2}{4}\right\rfloor\right].
  \end{gather}
  Define $f(a, b, c)$ as the function on the right-hand-side of (2.1). We show that $f(a, b, c)$
  attains its maximum $2\left\lfloor\frac{n^2}{4}\right\rfloor$ at $a = b = 0$ and $c = n$. For $b
  \geq 2$,
  \begin{align*}
    f(a, b - 2, c + 2) - f(a, b, c)
    &= \binom{a + c + 2}{2} - \binom{a + c}{2} \\
    &\qquad - 2\left[\binom{c + 2}{2} - \binom{c}{2} - \left\lfloor\frac{(c + 2)^2}{4}\right\rfloor + \left\lfloor\frac{c^2}{4}\right\rfloor\right] \\
    &= 2(a + c) + 1 - 2[2c + 1 - (c + 1)] \\
    &= 2a + 1 > 0.
  \end{align*}
  By symmetry, $f(a - 2, b, c + 2) > f(a, b, c)$, and thus $f$ attains its maximum when $c$ is $n -
  1$ or $n$, that is, $a + b \leq 1$. Equation (2.1) now yields, 
  \[
    e(G_1) + e(G_2) \leq 2\left\lfloor\frac{n^2}{4}\right\rfloor.
  \]
  Assume that $a = 0$. When $c = n$, the equality holds only if $G_1 = G_2 =
  K_{\left\lceil\frac{n}{2}\right\rceil, \left\lfloor\frac{n}{2}\right\rfloor}$. If $c = n - 1$,
  then the equality holds only if $n$ is odd, $G_1 = G[C] = K_{\left\lceil\frac{n -
  1}{2}\right\rceil, \left\lfloor\frac{n - 1}{2}\right\rfloor}$ and $G_2$ is a copy of $G_1$ with
  all vertices adjacent to the only remaining vertex, i.e. $G_2 = G_1 + K_1$.
\end{proof}

We now prove Theorem 2.1.

\begin{proof}[Proof of Theorem 2.1]
  Assume that $n > 1$. Put $G_{n + i} = G_i$. By Lemma 3.2,
  \[
    \sum_{i = 1}^n e(G_i) = \frac{1}{2}\sum_{i = 1}^n (e(G_i) + e(G_{i + 1})) \leq \frac{1}{2}\sum_{i = 1}^n 2\left\lfloor\frac{n^2}{4}\right\rfloor = n\left\lfloor\frac{n^2}{4}\right\rfloor.
  \]
  Suppose the equality holds. By Lemma 3.2, we are done if $n$ is even. Suppose $n$ is odd and $G_i
   = K_{\left\lceil\frac{n - 1}{2}\right\rceil, \left\lfloor\frac{n - 1}{2}\right\rfloor} + K_1$ for
   some $i$. By applying Lemma 3.2 repeatedly, 
   \begin{align*}
    G_i &= K_{\left\lceil\frac{n - 1}{2}\right\rceil, \left\lfloor\frac{n - 1}{2}\right\rfloor} + K_1 \\
    G_{i + 1} &= K_{\left\lceil\frac{n - 1}{2}\right\rceil, \left\lfloor\frac{n - 1}{2}\right\rfloor} \\
    G_{i + 2} &= K_{\left\lceil\frac{n - 1}{2}\right\rceil, \left\lfloor\frac{n - 1}{2}\right\rfloor} + K_1 \\
    &\qquad\quad \vdots
   \end{align*}
   and the alternation proceeds. But then $G_{n + i} = G_i =
   K_{\left\lceil\frac{n - 1}{2}\right\rceil, \left\lfloor\frac{n - 1}{2}\right\rfloor}$ as $n$ is
   odd, and this contradiction completes the proof.
\end{proof}

\section{Non-bipartite $F$}

\begin{theorem}
  Let $F$ be $(r + 1)$-colorable, with $r \geq 2$. Suppose that $E(G_i) \cap E(G_j)$ is $F$-free for
  distinct $i, j$. For large enough $n$,
  \[
    \sum_{i = 1}^n e(G_i) \leq n \cdot \textnormal{ex}(n, F),
  \]
  with equality if and only if $G_1 = G_2 = \cdots = G_n$ are $n$-vertex extremal graphs for $F$.
\end{theorem}

By the same argument as in Theorem 2.1, it suffices to prove the following lemma:

\begin{lemma}
  Let $F$ be $(r + 1)$-colorable, with $r \geq 2$. Suppose $E(G_1) \cap E(G_2)$ does not include
  $F$. For large enough $n$,
  \[
    e(G_1) + e(G_2) \leq 2 \cdot \textnormal{ex}(n, F),
  \]
  with equality if and only if $G_1 = G_2$ are $n$-vertex extremal graphs for $F$, unless $n$ is
  odd, $G_1$ is an $(n - 1)$-vertex extremal graph for $F$, and $G_2 = G_1 + K_1$. 
\end{lemma}

\begin{proof}
  Let $C = V(G_1) \cap V(G_2)$, $A = V(G_1) \backslash C$, and $B = V(G_2) \backslash C$. Put $a =
  |A|$, $b = |B|$, $c = |C|$. Since $G_1, G_2$ are induced graphs, $E(G_1[C]) =
  E(G_2[C]) = E(G[C]) = E(G_i) \cap E(G_j)$, which is $F$-free. Hence, 
  \begin{gather}
    e(G_1) + e(G_2) \leq \binom{a + c}{2} +  \binom{b + c}{2} - 2\left[\binom{c}{2} - \textnormal{ex}(c, F)\right].
  \end{gather}
  Define $f(a, b, c)$ as the function on the right-hand-side of (2.2). We show that $f(a, b, c)$
  attains its maximum at $a = b = 0$ and $c = n$. 

  \begin{claim}
    If $c \leq \frac{n}{2}$, then $f(a, b, c) < 2 \cdot \textnormal{ex}(n, F)$.
  \end{claim}

  \begin{proof}
    Write $c = kn$ for some $k \in [0, 1/2]$. Since
    \[
      f(a, b, kn) \leq 2\binom{(1 - k)n/2}{2} - 2\left[\binom{kn}{2} - \textnormal{ex}(kn, F)\right],
    \]
    it suffices to show that
    \[
      \textnormal{ex}(n, F) - \textnormal{ex}(c, F) > \binom{(1 - k)n/2}{2} - \binom{kn}{2}
    \]
    for all $k \in [0, 1/2]$. By the Erdős-Stone theorem, $\textnormal{ex}(n, F) = \left(1 -
    \frac{1}{r}\right)\frac{n^2}{2} + o(n^2)$ and so the left-hand-side is at least
    \[
      \textnormal{ex}(n, F) - \textnormal{ex}(c, F) \geq \textnormal{ex}(n, F) - \textnormal{ex}(n/2, F) \geq \left(1 - \frac{1}{r}\right)\left(\frac{n^2}{2} - \frac{n^2}{8}\right) - o(n^2) \geq \frac{3n^2}{16} - o(n^2)
    \]
    On the right-hand-side, 
    \[
      \binom{(1 - k)n/2}{2} - \binom{kn}{2} = (1 - 2k - 3k^2)\frac{n^2}{8} + o(n^2) \leq \frac{n^2}{8} + o(n^2).
    \]
    Combining the above inequalities now yields the claim, as $n$ is large.
  \end{proof}
  Thus we may assume that $c > \frac{n}{2}$. A theorem of Simonovits states that for large enough
  $n$, $\text{ex}(n, F) = \text{ex}(n, K_{r + 1}) + \text{ex}(n, \tilde{F})$, where $\tilde{F}$ is
  the family of residue subgraphs of $F$ after $F$ is embedded into $T_r(n)$. This implies 
  \[
    \text{ex}(n + 1, F) - \text{ex}(n, F) \geq \text{ex}(n, K_{r + 1}) - \text{ex}(n + 1, K_{r + 1}),
  \]
  and so
  \[
    f(a, b - 2, c + 2) - f(a, b, c) \geq 2a - 2c - 1 + 2[\textnormal{ex}(c + 2, K_{r + 1}) - \textnormal{ex}(c, K_{r + 1})].
  \]
  Since $\textnormal{ex}(c + 1, K_{r + 1}) - \textnormal{ex}(c, K_{r + 1}) \geq c - \left\lfloor
  \frac{c}{r} \right\rfloor \geq c - \left\lfloor \frac{c}{2} \right\rfloor$, we have
  \[
    \textnormal{ex}(c + 2, K_{r + 1}) - \textnormal{ex}(c, K_{r + 1}) \geq c + 1 - \left\lfloor \frac{c + 1}{2} \right\rfloor + c - \left\lfloor \frac{c}{2} \right\rfloor = c + 1,
  \]
  and thus 
  \[
    f(a, b - 2, c + 2) - f(a, b, c) \geq 2a + 1 > 0.
  \]

  By symmetry, we also have $f(a - 2, b, c + 2) > f(a, b, c)$. Thus, $f$ attains its maximum when
  $c$ is $n - 1$ or $n$. Equation (2.2) now yields, 
  \begin{align*}
    e(G_1) + e(G_2) \leq \max\left[2 \cdot \textnormal{ex}(n, F), 2 \cdot \textnormal{ex}(n - 1, F) + n - 1\right].
  \end{align*}
  Assume that $a = 0$. Since
  \begin{align}
    2 \cdot \textnormal{ex}(n, F) - 
    [2 \cdot \textnormal{ex}(n - 1, F) &+ n - 1] \nonumber \\
    &\geq 2[\textnormal{ex}(n, K_{r + 1}) - \textnormal{ex}(n - 1, K_{r + 1})] - n + 1 \\
    &= 2\left(n - \left\lceil \frac{n}{r} \right\rceil\right) - n + 1 \nonumber  \\
    &\geq n + 1 - 2\left\lceil \frac{n}{2} \right\rceil \geq 0,  \nonumber 
  \end{align}
  we have
  \[
    e(G_1) + e(G_2) \leq 2 \cdot \textnormal{ex}(n, F).
  \]
  If $c = n$, the equality holds only if $G_1 = G_2$ are $n$-vertex extramal graphs for $F$. Suppose
  $c = n - 1$ and the equality holds. Observe that equation (2.3) equals to zero only when $r = 2$
  and $n$ is odd. Hence, if $c = n - 1$, equality could only be achieved when $r = 2$, $n$ is odd,
  $G_1$ is an $(n - 1)$-vertex extremal graph for $F$, and $G_2 = G_1 + K_1$.
\end{proof}

\chapter{General Case}

We now relax the assumption that $G_1, \dots, G_m$ are induced subgraphs. The trivial construction
of putting $G_1 = K_n$ and $G_2, \ldots, G_m$ to be extremal graphs for $F$ yields the lower bound
\begin{gather}
	\sum_{i = 1}^m e(G_i) = \binom{n}{2} + (m - 1)\text{ex}(n, F).
\end{gather}
In this section we examine whether this bound is tight. The following is an asymptotic result on the
number of $G_i$'s:

\begin{theorem}
	Suppose that $E(G_i) \cap E(G_j)$ does not include $r$-vertex graph $F$ for distinct $i, j$. Then
	for large enough $n$,
	\[
		\sum_{i = 1}^m e(G_i) \leq m(1 + o(1))\textnormal{ex}(n, F),
	\]
	as $m \to \infty$.
\end{theorem}

\begin{proof}
  Pick $\epsilon > 0$. Reorder $G_1, \ldots, G_m$ so that $G_1, \ldots, G_{m'}$ are all the $G_i$'s
  containing at least $(1 + \epsilon)\text{ex}(n, F)$ edges. A theorem of Simonovits states that $G$
  contains at least $\delta n^r$ copies of $F$ for some $\delta = \delta(\epsilon)$. Since there can
  be at most $\binom{n}{r}$ copies of $F$ across all $G_i$'s, 
  \[
    m'\delta n^{r} \leq \binom{n}{r} \leq n^r \implies m' \leq \frac{1}{\delta}.
  \]
  It now follows that
  \begin{align*}
    \sum_{i = 1}^m e(G_i) 
    &= \sum_{i = 1}^{m'} e(G_i) + \sum_{i = m' + 1}^m e(G_i) \\
    &\leq \frac{1}{\delta}\binom{n}{2} + \left(m - \frac{1}{\delta}\right)(1 + \epsilon)\text{ex}(n, F) \\
    &= m\left[1 + \epsilon + \frac{1}{m\delta}\left(\frac{\binom{n}{2}}{\textnormal{ex}(n, F)} - (1 + \epsilon)\right)\right]\textnormal{ex}(n, F).
  \end{align*}
  Since $\epsilon$ is arbitrary, the result follows.
\end{proof}


\section{Triangle $F$}

Consider $F$ to be a triangle. Simply counting the number of triangles in each $G_i$ shows the
following:

\begin{theorem}
	For any $\epsilon > 0$, if $E(G_i) \cap E(G_j)$ does not include $K_3$ for distinct $i, j$, then
	\[
		\sum_{i = 1}^m  e(G_i) < m(1 + \epsilon)\frac{n^2}{4} + (1 - \epsilon)\frac{n^2}{2\epsilon}.
	\]
\end{theorem}

\begin{claim}
	There are less than $\frac{2}{\epsilon}$ number of $G_i$'s with $e(G_i) \geq (1 +
	\epsilon)\frac{n^2}{4}$.
\end{claim}

\begin{proof}
	Suppose $e(G_i) \geq (1 + \epsilon)\frac{n^2}{4}$ for $1\leq i \leq k$. Let $K_3(G)$ denote the
	number of triangles in graph $G$. By the Moon-Moser inequality,
	\[
		K_3(G_i) \geq \epsilon(1 + \epsilon)\frac{n^3}{12}.
	\]
	Since there are no overlapping traingles from different $G_i$'s, 
	\[
		\binom{n}{3} \geq \sum_{i = 1}^k K_3(G_i) \geq \frac{\epsilon(1 + \epsilon)}{12}kn^3.
	\]
	Rearranging yields $k < \frac{2}{\epsilon}$.
\end{proof}

By the claim, 
\[
  \sum_{i = 1}^m e(G_i) < \frac{2}{\epsilon}\binom{n}{2} + \left(m - \frac{2}{\epsilon}\right)(1 + \epsilon)\frac{n^2}{4} \leq m(1 + \epsilon)\frac{n^2}{4} + (1 - \epsilon)\frac{n^2}{2\epsilon},
\]
which proves Theorem 3.2.

It can be easily shown that the bound in Theorem 3.2 is tight when $m = 2$, as
\[
	e(G_1) + e(G_2) \leq \binom{n}{2} + e(G_{1, 2}) \leq \binom{n}{2} + \left\lfloor \frac{n^2}{4} \right\rfloor.
\]
This result is also true for $m = 3$:

\begin{proposition}
	Let $G_1, G_2, G_3$ be subgraphs of some $G$ such that no triangle is contained in any two graphs,
	then
	\[
		e(G_1) + e(G_2) + e(G_3) \leq \binom{n}{2} + \frac{n^2}{2}.
	\]
\end{proposition}

\begin{proof}
  Define $H_k \subseteq G$ be the graph with edges contained in at least $k$ number of $G_i$'s and
  note that $e(G_1) + e(G_2) + e(G_3) = e(H_1) + e(H_2) + e(H_3)$. Thus it suffices to show that
  $e(H_2) + e(H_3) \leq \frac{n^2}{2}$. Notice $H_2$ must not contain any triangles with two edges
  in $H_3$, so
  \[
    e(H_2) + e(H_3) \leq \binom{n}{2} + e(H_3) - |\{\{u, v\} : u \neq v, N_{H_3}(u) \cap N_{H_3}(v) \neq \emptyset\}|.
  \]
  Let $H'_3$ be the graph with the same vertex set as $H_3$ and edge set $\{\{u, v\} : u \neq v,
  N_{H_3}(u) \cap N_{H_3}(v) \neq \emptyset\}$. It suffices to show that $\frac{n}{2} \geq e(H_3) -
  e(H'_3)$. 

  Let $d_1 \geq d_2 \geq \cdots \geq d_n$ and $f_1 \geq f_2 \geq \cdots \geq f_n$ each be the degree
  sequence of $H_3$ and $H_3'$, respectively. We show that $f_i \geq d_i - 1$ for all $i$. Let $v_i$
  denote the vertex in $H$ with degree $d_i$ and $u_i$ be the vertex in $H$ with degree $f_i$. Let
  $S_i = |N_{H_3}(v_1) \cup \cdots \cup N_{H_3}(v_i)|$. Since 
  \[
    \sum_{u \in S_i} d_{H_3}(u) \geq d_1 + \cdots + d_i,
  \]
  we have that $|S_i| \geq i$. But then $S_i \backslash \{u_1, \ldots, u_{i - 1}\}$ is non-empty,
  and every $u \in S_i$ has degree $d_{H_3'}(u) \geq d_i - 1$. Hence, $f_i \geq d_i - 1$ for all
  $i$, which yields
  \[
    e(H_3') = \frac{1}{2}\sum_{i = 1}^n f_i \geq \frac{1}{2}\sum_{i = 1}^n (d_i - 1) = e(H_3) - \frac{n}{2}.
  \]
\end{proof}

However, the bound in Proposition 3.1 is not tight for $m \geq 4$, as shown in the following graph:

\begin{center}
  \begin{tikzpicture}
    \draw (4, 2) to node[midway, above] {$123$} (-4, 2);
    \draw (-4, 2) to node[midway, left] {$124$} (-1, -1);
    \draw (-1, -1) to node[midway, above] {$34$} (1, -1);
    \draw (1, -1) to node[midway, right] {$124$} (4, 2);
    \draw (0, 1) to node[midway, below] {$234$} (4, 2);
    \draw (0, 1) to node[midway, below] {$134$} (-4, 2);
    \draw (0, 1) to node[midway, left] {$123$} (-1, -1);
    \draw (0, 1) to node[midway, right] {$123$} (1, -1);
    \draw[bend left=80] (4, 2) to node[midway, right] {$134$} (-1, -1);
    \draw[bend right=80] (-4, 2) to node[midway, left] {$234$} (1, -1);

    \foreach \x/\y in {0/1, 4/2, 1/-1, -1/-1, -4/2} {
        \fill (\x, \y) circle (2pt);
    }
  \end{tikzpicture}
  \\
  \small{The number on each edge denotes the set of $G_i$'s that contain the edge.}
\end{center}
The above graph contains $29$ edges, which exceeds the bound $\binom{5}{2} + 3 \lfloor \frac{5^2}{4}
\rfloor = 28$ by $1$. By blowing up the above graph, we can construct a graph with $n \in
10\mathbb{Z}$ vertices that contains
\[
  5\binom{n/5}{2} + 29 \cdot \frac{(n/5)^2}{4}
\]
edges, which exceeds the bound $\binom{n}{2} + 3\lfloor\frac{n^2}{4}\rfloor$ by $n^2/100$. 

\section{Bipartite $F$}

In this section, we discuss the case where $F$ is bipartite. In particular, we focus on the cases
where $F \subseteq K_{2, 2}$ is $P_2$, a path of length $2$, or $M_2$, a matching with two edges.

\begin{theorem}
  Given graph $G$ and $G_1, \cdots, G_n$ subgraphs of $G$ with $E(G_i) \cap E(G_j)$ not containing
  $P_2$ for distinct $i, j$. Then
  \[
    \sum_{i = 1}^m e(G_i) \leq \frac{n^{5/2}}{2}.
  \]
\end{theorem}

\begin{proof}
  Since there are no overlapping $P_2$'s in different $G_i$'s, 
  \[
    \sum_{i = 1}^n \#\{P_2 \subseteq G_i\} \leq \#\{P_2 \subseteq G\}
  \]
  For each $G_i$, each vertex $v$ in $G_i$ and two of its neighbors form one unique $P_2$, so
  \[
    \#\{P_2 \subseteq G_i\} = \sum_{v \in V(G_i)} \binom{d_{G_i}(v)}{2}.
  \]
  And by Jensen's inequality,
  \[
    \sum_{v \in V(G_i)} \binom{d_{G_i}(v)}{2} \geq n\binom{d_{G_i}(v)/n}{2} = n\binom{2e(G_i)/n}{2} \geq \frac{2(e(G_i))^2}{n}.
  \]
  On the other hand, since each three vertices in $G$ can form at most three $P_2$'s, 
  \[
    \#\{P_2 \subseteq G\} \leq 3\binom{n}{3} = \frac{n^3}{2}.
  \]
  Combining the above inequalities yields
  \[
    \sum_{i = 1}^n \frac{2(e(G_i))^2}{n} \leq \frac{n^3}{2} \overset{Jensen's}{\implies} \sum_{i = 1}^n e(G_i) \leq \frac{n^{5/2}}{2}.
  \]
\end{proof}

\textcolor{red}{TODO: add the projective plane construction.}

\begin{theorem}
  Given graph $G$ and $G_1, \cdots, G_n$ subgraphs of $G$ with $E(G_i) \cap E(G_j)$ not containing
  $M_2$ for distinct $i, j$. Then
  \[
    \sum_{i = 1}^n e(G_i) \leq n^{5/2}.
  \]
\end{theorem}

\begin{proof}
  Notice that $\#\{M_2 \subseteq G\} = \binom{e(G_i)}{2}$. On the other hand, each four vertices in
  $G$ can form at most three $M_2$'s, so $\#\{M_2 \subseteq G\} \leq 3\binom{n}{4} \leq
  \frac{n^4}{8}$. By the same argument as in Theorem 3.4, we have 
  \[
    \sum_{i = 1}^n \binom{e(G_i)}{2} = \sum_{i = 1}^n \#\{M_2 \subseteq G_i\} \leq \#\{M_2 \subseteq G\} \leq \frac{n^4}{8}.
  \]
  By Jensen's inequality,
  \[
    \sum_{i = 1}^n \binom{e(G_i)}{2} \geq n\binom{\sum_{i = 1}^n e(G_i)/n}{2} = \frac{1}{2n}\left[\left(\sum_{i = 1}^n e(G_i)\right)^2 - n\sum_{i = 1}^n e(G_i)\right].
  \]
  Combining the above inequalities yields
  \[
    \left(\sum_{i = 1}^n e(G_i)\right)^2 - n\sum_{i = 1}^n e(G_i) \leq \frac{n^5}{4},
  \]
  and solving the quadratic inequality gives
  \[
    \sum_{i = 1}^n e(G_i) \leq n^{5/2}.
  \]
\end{proof}

We may obtain a tight bound if we forbid both $P_2$ and $M_2$ at the same time:

\begin{theorem}
  Given graphs $G_1, \ldots, G_n \subseteq K_n$ with $E(G_i) \cap E(G_j)$ not containing $P_2$ or
  $M_2$ for distinct $i, j$. Then
  \[
    \sum_{i = 1}^n e(G_i) \leq n^2 - n,
  \]
  and this bound is tight.
\end{theorem}

\begin{proof}
  Denote the set of $G_i$'s as $\{G_i\} = \{G_1, \ldots, G_n\}$, and the set of distinct pairs of
  $G_i$'s as $\{G_i\}^2 = \{\{G_j, G_k\} : j \neq k\}$. Consider the bipartite graph $H$ with
  vertex set $V(H) = \{G_i\} \sqcup E(K_n)$ and edge set $E(H) = \{\{G_j, e\} \in \{G_i\} \times
  E(K_n) : e \in G_j\}$. Define $\phi: \{G_i\}^2 \to 2^{E(K_n)}$ by sending each $\{G_j, G_k\}$ to
  their common edge set $E(G_j) \cap E(G_k)$. Notice that each distinct $G_j, G_k$ have at most
  one edge in common, so $|\phi(G_j, G_k)| \leq 1$. On the other hand, each edge $e \in E(G)$ can
  be obtained via $\phi$ by $\binom{d_H(e)}{2}$ possible distinct pairs $(G_j, G_k)$, and thus
  $|\phi^{-1}(e)| = \binom{d_H(e)}{2}$. But then
  \[
    \binom{n}{2} \geq \sum_{(G_j, G_k) \in \{G_i\}^2} |\phi(G_j, G_k)| = \sum_{e \in E(K_n)} |\phi^{-1}(e)| = \sum_{e \in E(K_n)} \binom{d_H(e)}{2}.
  \]
  By Jensen's inequality,
  \[
    \sum_{e \in E(K_n)} \binom{d_H(e)}{2} \geq \binom{n}{2}\binom{\sum_{e \in E(K_n)} d_H(e)/\binom{n}{2}}{2} = \binom{n}{2}\binom{\sum_{i = 1}^n e(G_i)/\binom{n}{2}}{2}.
  \]
  Combining the above inequalities yields
  \[
    2\binom{n}{2}^2 \geq \left(\sum_{i = 1}^n e(G_i)\right)^2 - \binom{n}{2}\sum_{i = 1}^n e(G_i),
  \]
  and the result now follows from solving the quadratic inequality.

  To see that this bound is tight, consider the construction such that for each distinct $i, j \in
  [n]$, $E(G_i) \cap E(G_j)$ contains exactly one unique edge $e \in K_n$. The number of edges in
  this construction is $2\binom{n}{2} = n^2 - n$.
\end{proof}

 

\begin{theorem}
  Let $F$ be a graph such that there exists extremal $F$-free subgraph $H$ on $n$ vertices that have
  maximum degree $\Delta(H) = o(n^{1/2})$. For large enough $n$, if $G_1, \ldots, G_m$ are graphs on
  common vertex set $[n]$ such that no copy of $F$ is contained in any $k$ of the $G_i$'s, then
  \[
    \sum_{i = 1}^m e(G_i) = (k - 1)\binom{n}{2} + \textnormal{ex}(n, F)\binom{m}{k},
  \]
  and the bound is tight.
\end{theorem}

\begin{proof}
  For $S \subseteq [m]$, let $E_S$ denote the set of edges that are contained in exactly $\{G_i\}_{i
  \in S}$. Then 
  \[
    \sum_{i = 1}^m e(G_i) = \sum_{S \subseteq [m]} |S||E_S| \leq (k - 1)\binom{n}{2} + \sum_{S \subseteq [m], |S| \geq k} (|S| - k + 1)|E_S|.
  \]
  Let $A_S = \bigcup_{T \supseteq S} E_T$, i.e. the set of edges that are contained in all $G_i$
  with $i \in S$. When $|S| \geq k$, the edge set $A_S$ is $F$-free and thus 
  \[
    \sum_{T \supseteq S} |E_T| \leq \text{ex}(n, F).
  \]
  Hence,
  \[
    \sum_{\substack{S \subseteq [m] \\ |S| \geq k}} (|S| - k + 1)|E_S| = \sum_{\substack{S \subseteq [m], \\ |S| = k}} \sum_{T \subseteq S} \frac{(|T| - k + 1)|E_T|}{\binom{|T|}{k}} \leq \sum_{\substack{S \subseteq [m], \\ |S| = k}} \sum_{T \subseteq S} |E_T| \leq \binom{m}{k}\text{ex}(n, F),
  \]
  as each $T \in [m]$ with $|T| \geq k$ is counted $\binom{|T|}{k}$ times in total and $|T| - k + 1
  \leq \binom{|T|}{k}$. This proves the upper bound.

  Now we show the bound is tight. In particular, we need to show there exists a construction such
  that the graph with edge set $E_S$ is an extremal $F$-free graph, for all $S \subseteq [m]$ of
  size $k$. Let $M = \binom{m}{k}$ and $H_1, \ldots, H_M$ be copies of an extremal $F$-free graph on
  $n$ vertices. It suffices to show that we can embed each $H_i$ onto $[n]$ such that their edge
  sets are pairwise disjoint. We begin by an arbitrary embedding of each $H_i$ and iteratively
  decrease the number of intersecting edges. Define a $(u, v, i)$-swap by swapping the embedding of
  vertex $u$ and $v$ of $H_i$, i.e. replacing each edge $\{u, w\} \in E(H_i)$ with the edge $\{u,
  w\}$ and each edge $\{v, w\} \in E(H_i)$ with the edge $\{v, w\}$. This perserves the type of
  isomorphism of $H_i$. Given a vertex $v$, let $N(v) = N_{H_1}(v) \cup \cdots \cup N_{H_M}(v)$.
  Suppose there exists an intersecting edge $\{u, w\} \in E(H_i) \cap E(H_j)$. Since the maximum
  degree of each $H_i$ is $o(n^{1/2})$, $|N(u) \cup N(N(u))| = o(n)$ so there exists a vertex $v
  \notin N(u) \cup N(N(u))$. Since $N(u) \cap N(v) = \emptyset$, performing a $(u, v, i)$-swap
  reduces the number of intersecting edges. The result now follows by iterating this process.
\end{proof}


\end{document}
